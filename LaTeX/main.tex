\documentclass{report} %fiks nog tekstgrootte, font, etc.
\usepackage{graphicx} % Required for inserting images
\usepackage[left=30mm, top=20mm]{geometry}
\usepackage[dutch]{babel}
\title{PWS Cryptografie - fiks titel nog!}
\author{Otto Crawford (6A), Christiaan Tjong Tjin Tai (6B)}
\date{\today}
% We moeten nog vak en begeleider erin gooien maar komt goed (en ook datum in NL fiksen)
\begin{document}

\maketitle

\tableofcontents

\chapter{Voorwoord}

\chapter{Samenvatting}
\section{Samenvatting in het Nederlands}
\section{Samenvatting in het Engels}


\chapter{Inleiding}

%evt nog \part{Het maken}?
\chapter{Wat is cryptografie?}
%Voelt als een soort inleiding - kan ook daarbij worden gezet; fiks nog aub!

\chapter{Wiskunde in de cryptografie}
Er zijn een aantal onderdelen in de wiskunde die erg belangrijk zijn in de cryptografie. Met name wiskunde op het gebied van priemgetallen en modulair rekenen is erg belangrijk. Hieronder volgen een aantal definities die wij zullen gebruiken.
\section{Priemgetallen}

\section{Grootste gemene deler}

\section{Modulair rekenen}

\section{Inverse}

\chapter{Verschillende algoritmes}
\section{Priemgetallen genereren}

\section{Diffie-Hellman???}

\section{RSA}

\section{Digital signatures}

\chapter{Een eigen protocol}

\section{Vereisten}

\section{Implementatie}

%evt nog \part{Het kraken}?
\chapter{Cybercriminaliteit}

\chapter{Quantumcomputers}
\section{Gevaren}
%Ook gevaren voor eigen protocol benoemen!

\section{Oplossingen}

\chapter{Conclusie}

\chapter{Discussie en reflectie}

\chapter{***fiks bibliografie nog***}
%\bibliography{}

\chapter{Logboek}
\section{Otto}

\section{Christiaan}

%nog code invoegen???

\chapter{Verklaring eigen werk}

\end{document}
